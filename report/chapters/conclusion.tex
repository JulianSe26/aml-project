% !TEX root =  master.tex
\chapter{Conclusion}\label{chapter:conclusion}
\chapterauthor{Written by All}

In this project we wanted to create a \ac{CAD} like system that is capable of detecting and classifying COVID-19 infections based on \acl{CXR} images by solving the eponymous Kaeggle challenge. Our solution consists of a multi-stage approach including several deep neural networks to achieve the goal set by the challenge creators. We showed that a \ac{YOLO} and a Faster \ac{R-CNN} are suitable candidates for detecting and localizing COVID-19 regions of interests. Since the dataset required by the challenge is limited, we decided to include the \ac{RSNA} and \ac{NIH} data from previous challenges to pre-train our model weights.
We experienced that the Faster \ac{R-CNN} outperforms the \ac{YOLO} model on our test set, which is also consistent with other work in the field of object detection. However, for our final COVID-19 detection, we created an ensemble model that uses a weighted box fusion of both model outputs ultimately leading to the best results with a \ac{mAP}@$.5$ score of $0.451$. In general we are pleased with the final outcome of the COVID-19 detection performance and think that our work can be used as a solid basis for further research in this area. Due to the limited data we saw some overfitting effects with increased training time, which may be a starting point for further research. This may involve a repetition of our experiments with more data, an ablation study on how the pre-training of both models contribute to the performance or may be a root cause for overfitting effects, a hyper-parameter optimization process or even including more models with different architectures into the ensemble combination. Natural candidates would be for example a RetinaNet \autocite{lin_focal_2018} or a segmentation-based mask R-CNN \autocite{maskRcnn}.
Another interesting extension may include research about an end-to-end train process of the ensemble approach, meaning that there could be an improvement in performance if both single models would be trained together in a joint fashion. However, we only had limited compute resources and time budget which is why we leave this open for further research.

TORBENS CONCLUSION HERE

rescale images to 8 bit w.r.t. tissue/bones (..)

In this report we have talked a lot about how different machine learning methods, and in this case computer vision, can help combat the pandemic. Honorably ML researchers and doctors all around the world scrambled to try and help at the start and during the pandemic by providing datasets, studies and models concerning all facets of what to do against the virus, for the infected or for society as a whole. Recently though, doubts have been cast on these efforts, as some studies suggest that machine learning models might not be very effective or in the worst case even dangerous. In a large scale meta study, published in the British Medical Journal, \citeauthor{wynants_prediction_2020} analyze 232 models proposed in 169 studies and find that not a single one is suitable for clinical use but that at least two might be promising with more work \autocite{wynants_prediction_2020}. Another study by \citeauthor{aix-covnet_common_2021} that closely examines 62 studies regarding the topic that we were also working on of identifying COVID-19 infections in patients, comes to the same conclusion that none of them are suitable for clinical use \autocite{aix-covnet_common_2021}. These concerns show that while there has been great effort and a lot of good will in the research community, the processes and studies are not yet as matured as maybe ML researchers and practitioners would like. While most times no harm is done in other areas of ML usage when an error is made, the stakes in the medical field are a lot higher and therefore a lot more care has to be put into every study and ML model that is proposed.

All in all, we liked working on this project. We think that with the global situation around COVID-19, ML may be a key-factor in solving different issues regarding medical image analysis while still keeping the aforementioned concerns in mind. Overall we are very pleased with our results, also achieving all goals set by the underlying Kaeggle Challenge.

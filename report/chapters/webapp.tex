% !TEX root =  master.tex
\chapter{Proposed web application}\label{chapter:webapp}
\chapterauthor{Written by Tobias Richstein}

Since we do not plan to hand in a sample notebook to the Kaeggle challenge, we decided to build some sort of application that makes the result of our work a little more accessible and easily usable. In the end we settled on providing access to the models via a website that runs inference on the backend. We settled on using the framework \textit{Flask} \autocite{ronacher_flask_nodate} since it uses Python just as the rest of the project.

The interface is very simple but functional and built with the templating engine \textit{Jinja 2} that is included with Flask. The user can first upload an image that is then previewed before choosing which model configuration to run: Faster R-CNN, Yolo or the ensemble of both. When clicking the submit button inference is run on the backend and the result is returned by re-rendering the webpage with the bounding boxes that were predicted. Their scores and coordinates also appear alongside the image. 

We also provide a Dockerfile to containerize the entire application including the models to easily be able to build and run our results anywhere. Additionally a prebuilt Docker image is available on Docker Hub and can be run with: \texttt{docker run -p 5000:5000 tobiasrst/aml-project:latest}. The app will then be available on Port 5000. Using the sample image located in the project directory at \texttt{/src/app/example.png} the inference can be tested quickly.
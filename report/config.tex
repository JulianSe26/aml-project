% !TEX root =  master.tex
% 		HYPERREF
%
\usepackage[
hidelinks=true % keine roten Markierungen bei Links
]{hyperref}

% \newcommand*{\Kapitel}{../Kapitel}


\newcommand{\TitelDerArbeit}[1]{\def\DerTitelDerArbeit{#1}\hypersetup{pdftitle={#1}}}
\newcommand{\AutorDerArbeit}[1]{\def\DerAutorDerArbeit{#1}\hypersetup{pdfauthor={#1}}}
\newcommand{\ZweiterAutorDerArbeit}[1]{\def\DerZweiteAutorDerArbeit{#1}\hypersetup{pdfauthor={#1}}}
\newcommand{\DritterAutorDerArbeit}[1]{\def\DerDritteAutorDerArbeit{#1}\hypersetup{pdfauthor={#1}}}
\newcommand{\Firma}[1]{\def\DerNameDerFirma{#1}}
\newcommand{\Kurs}[1]{\def\DieKursbezeichnung{#1}}



%		FONT AND INPUT ENCODING
%
\usepackage[T1]{fontenc}
\usepackage[utf8]{inputenc}

%		CALCULATIONS
%
\usepackage{calc} % Used for extra space below footsepline

%		LANGUAGE SETTINGS
%
%\usepackage[ngerman]{babel} 	% German language
%\usepackage[german=quotes]{csquotes} 	% correct quotes using \enquote{}

\usepackage[english]{babel}   % For english language
\usepackage{csquotes} 	% Richtiges Setzen der Anführungszeichen mit \enquote{}


%		BIBLIOGRAPHY SETTINGS
%
%\usepackage[backend=biber, autocite=footnote, style=authortitle-ibid, dashed=false,ibidtracker=context]{biblatex} 	%Use Author-Year-Cites with footnotes
 \usepackage[backend=biber, autocite=inline, style=ieee, , sorting=nty, dashed=false]{biblatex} 	% Use IEEE-Style (e.g. [1])
% \usepackage[backend=biber, autocite=inline, style=alphabetic]{biblatex} 	% Use alphabetic style (e.g. [TGK12])
%%%% APA/Harvard-Style (bitte die nächten zwei Zeilen auskommentieren)
%\usepackage[backend=biber, style=apa]{biblatex} 	
%\DeclareLanguageMapping{german}{german-apa}

\DefineBibliographyStrings{ngerman}{  %Change u.a. to et al. (german only!)
	andothers = {{et\,al\adddot}},
}
\DefineBibliographyStrings{ngerman}{%
	urlseen = {Last accessed},
}

%%% Uncomment the following lines to support hard URL breaks in bibliography 
%\apptocmd{\UrlBreaks}{\do\f\do\m}{}{}
%\setcounter{biburllcpenalty}{9000}% Kleinbuchstaben
%\setcounter{biburlucpenalty}{9000}% Großbuchstaben


\setlength{\bibparsep}{\parskip}		%add some space between biblatex entries in the bibliography
\addbibresource{bibliography.bib}	%Add file bibliography.bib as biblatex resource


%		FOOTNOTES 
%
% Count footnotes over chapters
\usepackage{chngcntr}
\counterwithout{footnote}{chapter}


%	ACRONYMS
%%%
%%% WICHTIG: Installieren Sie das neueste Acronyms-Paket!!!
%%%
\makeatletter
\usepackage[nolist]{acronym}
\@ifpackagelater{acronym}{2015/03/20}
{%
	\renewcommand*{\aclabelfont}[1]{\textbf{\textsf{\acsfont{#1}}}}
}%
{%
}%
\makeatother

%		LISTINGS
\usepackage{listings}	%Format Listings properly
\usepackage{xcolor}
\usepackage{bera}% optional: just to have a nice mono-spaced font
%\renewcommand{\lstlistingname}{Quelltext} 
\renewcommand{\lstlistlistingname}{Quelltextverzeichnis}

\lstset{numbers=left,
	numberstyle=\tiny,
	captionpos=b,
	basicstyle=\ttfamily\small}

\colorlet{punct}{red!60!black}
\definecolor{background}{HTML}{EEEEEE}
\definecolor{delim}{RGB}{20,105,176}
\colorlet{numb}{magenta!60!black}

\lstdefinelanguage{json}{
	basicstyle=\normalfont\ttfamily,
	numbers=left,
	numberstyle=\scriptsize,
	stepnumber=1,
	numbersep=8pt,
	showstringspaces=false,
	breaklines=true,
	frame=lines,
	backgroundcolor=\color{background},
	literate=
	*{0}{{{\color{numb}0}}}{1}
	{1}{{{\color{numb}1}}}{1}
	{2}{{{\color{numb}2}}}{1}
	{3}{{{\color{numb}3}}}{1}
	{4}{{{\color{numb}4}}}{1}
	{5}{{{\color{numb}5}}}{1}
	{6}{{{\color{numb}6}}}{1}
	{7}{{{\color{numb}7}}}{1}
	{8}{{{\color{numb}8}}}{1}
	{9}{{{\color{numb}9}}}{1}
	{:}{{{\color{punct}{:}}}}{1}
	{,}{{{\color{punct}{,}}}}{1}
	{\{}{{{\color{delim}{\{}}}}{1}
	{\}}{{{\color{delim}{\}}}}}{1}
	{[}{{{\color{delim}{[}}}}{1}
	{]}{{{\color{delim}{]}}}}{1},
}

%		EXTRA PACKAGES
\usepackage{lipsum}    %Blindtext
\usepackage{graphicx} % use various graphics formats
\usepackage[german]{varioref} 	% nicer references \vref
\usepackage{caption}	%better Captions
\captionsetup[table]{position=bottom} 
\usepackage{booktabs} %nicer Tabs
\usepackage{array}
\usepackage{pdfpages} 
\usepackage{algpseudocode}% http://ctan.org/pkg/algorithmicx
%\newcolumntype{P}[1]{>{\raggedright\arraybackslash}p{#1}}
\usepackage{tikz}
\usepackage{amssymb}

\usepackage{multirow}

\newcommand\MyBox[2]{
	\fbox{\lower0.75cm
		\vbox to 1.7cm{\vfil
			\hbox to 1.7cm{\hfil\parbox{1.4cm}{#1\\#2}\hfil}
			\vfil}%
	}%
}



%chapter settings
\usepackage{etoolbox}

%		ALGORITHMS
\usepackage{algorithm}
\usepackage{amsmath}
\usepackage{algpseudocode}
\usepackage{soul}
\usepackage{bchart}
\renewcommand{\listalgorithmname}{Algorithmenverzeichnis}
\floatname{algorithm}{Algorithmus}


%		FONT SELECTION: Entweder Latin Modern oder Times / Helvetica
\usepackage{lmodern} %Latin modern font
\usepackage{tabularx}
%\usepackage{mathptmx}  %Helvetica / Times New Roman fonts (2 lines)
%\usepackage[scaled=.92]{helvet} %Helvetica / Times New Roman fonts (2 lines)

%		PAGE HEADER / FOOTER
%	    Warning: There are some redefinitions throughout the master.tex-file!  DON'T CHANGE THESE REDEFINITIONS!
\RequirePackage{scrlfile}
\ReplacePackage{scrpage2}{scrlayer-scrpage}
\RequirePackage[automark,headsepline,footsepline]{scrpage2}
\pagestyle{scrheadings}
\renewcommand*{\pnumfont}{\upshape\sffamily}
\renewcommand*{\headfont}{\upshape\sffamily}
\renewcommand*{\footfont}{\upshape\sffamily}
\renewcommand{\chaptermarkformat}{}

\clearscrheadfoot

\ifoot[\rule{0pt}{\ht\strutbox+\dp\strutbox}Advanced Machine Learning]{\rule{0pt}{\ht\strutbox+\dp\strutbox}Advanced Machine Learning}
\ofoot[\rule{0pt}{\ht\strutbox+\dp\strutbox}\pagemark]{\rule{0pt}{\ht\strutbox+\dp\strutbox}\pagemark}

\ohead{\headmark}

\usepackage{chngcntr}
\counterwithout{figure}{chapter}
\counterwithout{table}{chapter}

%zeilenumbrüche
\setlength{\emergencystretch}{1em}

\usetikzlibrary{calc,trees,positioning,arrows,chains,shapes.geometric,%
	decorations.pathreplacing,decorations.pathmorphing,shapes,%
	matrix,shapes.symbols,quotes,angles,babel,calc}
\usepackage{tikz-qtree-compat,tikz-qtree}

\nocite{*}

\tikzset{
	>=stealth',
	punktchain/.style={
		rectangle, 
		rounded corners, 
		% fill=black!10,
		draw=black, very thick,
		text width=10em, 
		minimum height=3em, 
		text centered, 
		on chain},
	line/.style={draw, thick, <-},
	element/.style={
		tape,
		top color=white,
		bottom color=blue!50!black!60!,
		minimum width=8em,
		draw=blue!40!black!90, very thick,
		text width=10em, 
		minimum height=3.5em, 
		text centered, 
		on chain},
	every join/.style={->, thick,shorten >=1pt},
	decoration={brace},
	tuborg/.style={decorate},
	tubnode/.style={midway, right=2pt},
}


\definecolor{myblue}{RGB}{209, 216, 224}

\usepackage{amsthm}

\theoremstyle{plain}
\newtheorem{thm}{Theorem}[chapter] % reset theorem numbering for each chapter

\theoremstyle{definition}
\newtheorem{defn}[thm]{Definition} % definition numbers are dependent on theorem numbers
\newtheorem{exmp}[thm]{Example} % same for example numbers

\usepackage{epstopdf}

\makeatletter
\newcommand{\chapterauthor}[1]{%
	{\parindent0pt\vspace*{-65pt}%
		\linespread{1.1}\large\scshape#1%
		\par\nobreak\vspace*{5pt}}
	\@afterheading%
}
\makeatother

\makeatletter
\newcommand{\sectionauthor}[1]{%
	{\parindent0pt\vspace*{-20pt}%
		\linespread{1.1}\large\scshape#1%
		\par\nobreak\vspace*{5pt}}
	\@afterheading%
}
\makeatother

\makeatletter
\newcommand{\subsubsectionauthor}[1]{%
	{\parindent0pt\vspace*{-20pt}%
		\linespread{1.1}\scshape#1%
		\par\nobreak\vspace*{5pt}}
	\@afterheading%
}
\makeatother

	\subsubsection{Notes on the project structure and steps for reproducibility}
	
	The project source code is available in \textit{source} folder on root level of the submitted zip. This folder contains several jupyter notebooks where the most of our research done is implemented. In addition there are also some \textit{Python} files included for several tasks like downloading the dataset, utils etc.. The following table includes all files and their corresponding description.
	
	\begin{table*}[h!]
		\centering
		\begin{tabular}{lrl}
			\hline folder & filename & description \\ \hline
			./YOLO &  utils.py   &  Utils file for YOLO, including helper functions for processing YOLO. \\
			./YOLO & darknet\_moddel.py & Implementation of the YOLO darknet model in pytorch. \\
			./YOLO &  YOLO.ipyn & Jupyter notebook which includes data preparation and training of YOLO. \\
			 \hline
		\end{tabular}
		\caption{Project files structure.}
		\label{tab0}
	\end{table*}
	
	To reproduce our results, a couple of steps need to be executed for gathering the original data, the augmented data, preprocessing the data and finally training the models:
	\begin{enumerate}
		\item Download the annotated TACO images by executing the \textit{download.py} Python script.
		\item Download the augmented data from our file-sharing application using the \textit{xyz.py} Python script.
		\item Conversion of coco-format json to YOLO or RCNN specific format
		\item Compiling and running mobile application 
		\item wie sollen wir das tflite model providen ? oder allgemein die pretrained models 
		\item Soll ich die google service plist auch mitabegen ?
		\item For running the app on an emulator or on your personal device (make sure debugging is enabled) .... flutter clean flutter run usw.
	\end{enumerate}
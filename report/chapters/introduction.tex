% !TEX root =  master.tex
\chapter{Introduction and problem definition}\label{chapter:intro}

\section{Kaeggle Challenge}
\sectionauthor{Written by Torben Krieger}

\section{Related Work}\sectionauthor{Written by Julian Seibel}
Exploring the possibilities of Computer Vision methods in medical image analysis reaches back to the late 90's where first proposals of \ac{CAD} systems were designed like in \autocite{kraus2000aluminium}, where the authors detect aluminium dust-induced lung diseases.
With the rise of deep learning, medical image analysis on X-rays is subject of recent academic research reaching from detecting diseases like pneumonia \autocite{pneumoniaDetection} \autocite{pneumoniaDetection2} \autocite{gupta2019evolutionary} to tuberculosis and different thoracic diseases \autocite{jangam2021deep} and even pulmonary \autocite{vieira2021detecting}. Many of the contributions utilize a \ac{CNN} as single predictor. However there are interesting approaches using ensemble methods like in \autocite{livieris2019weighted}, where multiple predictors assigned with weights contribute to the final prediction. This stabilizes the training process and leads to a higher overall quality of the model by utilizing different advantages of different single models.

Even though it is not completely clear how a COVID-19 infection impacts the human body, it is possible to detect typical patterns in lungs using for example chest X-rays of affected patients. This opens many possibilities to provide fast and solid \ac{CAD} solutions while build on knowledge of previous works.
Since the pandemic started in 2020, there were many works published that deal with providing a reliable system for detecting COVID-19 infections using medical images. A \ac{CAD} based process would help the public health sector fighting the pandemic and would relieve medical personal in hospitals by increasing the automation of X-ray examination. Since this is of great importance for the society, several challenges were brought to life coming up with a wide variety of solutions. In the following, we will introduce some of them that we consider suitable for our approach.

The COVID-Net \autocite{wang2020covid} by \citeauthor{wang2020covid} uses a machine-driven exploration process to design a deep convolution learning model capable of classifying X-ray images of lungs into three categories (normal, pneumonia, COVID-19). The results of their study look promising and although the authors stated explicitly the non-production-ready state, the work can be used as a basis for future projects.

Another work using a deep neural network is the proposed CovidAID \autocite{mangal2020covidaid} network by \citeauthor{mangal2020covidaid}. In their approach, they use a pre-trained CheXNet \autocite{rajpurkar2017chexnet} \ac{CNN} while substituting the output layer of the model to fit it to their needs of predicting one of the four classes (normal, bacterial pneumonia, viral pneumonia, COVID-19). The authors applied transfer learning by only actively apply the train algorithm to the final layers of their proposed model, whereas the backbone weights are frozen.
In their final comparison, the authors reported to significantly improve upon the results on the previous introduced COVID-Net.
Similar to this, in the proposed method \autocite{CoronaDLTransfer} by \citeauthor{CoronaDLTransfer} transformation learning is applied to fine-tune and compare three state-of-the-art models (Inception ResNetV2, InceptionNetV3 and NASNetLarge) to detect COVID-19 infections, non-COVID-19 infections (like pneumonia) and no infection. Despite the high accuracy of 99\% achieved by the InceptionNetV3 model, the authors reported that all models suffer from overfitting due to limited amount of available data.

Since we are participating in the previously described challenge, we have a fixed requirement in terms of predicting not only if a X-ray image is COVID-19 positive, we rather need to detect and locate suspicious areas in such images. There exist a handful of related works we selected that try to achieve a similar goal, namely \autocite{brunese2020explainable}, \autocite{fan2020inf} and \autocite{al2021fast}.

Regarding the first, \citeauthor{brunese2020explainable} \autocite{brunese2020explainable} introduce a composed approach consisting of three different phases. In the first phase a image is classified as \enquote{potentially positive} by predicting if there is pneumonia patterns present in the image. The second phase then tries to differ the finding into COVID-19 caused or just pneumonia. In their last step, the authors designed a process using Gradient-weighted Class Activation Mapping to localize the affected areas that are symptomatic of a COVID-19 presence. With this approach the authors are able to add some explainability to their model by providing visually which areas are important for the final decision of the network.\newline
The second approach by \citeauthor{fan2020inf} \autocite{fan2020inf} proposed a model called Lung Infection Segmentation Deep Network (Inf-Net) that is capable of identifying and segmenting suspicious regions typical for COVID-19 infections. They use a partial decoder approach to generate a global representation of segmented maps followed by a implcit reverse attention and explicit edge attention mechanism to enhance the map boundaries. Due to the limitation of available data, the authors proposed a semi-supervised framework for training the model.\newline
The last approach by \citeauthor{al2021fast} \autocite{al2021fast} a \ac{YOLO} object detection model \autocite{yoloOriginal} is used to detect and diagnose COVID-19, being also capable of differentiating it from eight other respiratory diseases. In contrast to the former approach, the proposed \ac{CAD} system using the \ac{YOLO} model classifies and predicts bounding boxes for regions of interest. In their paper, the authors reported a diagnostic accuracy of 97.40\%.




\section{Proposed Solution}
\sectionauthor{Written by Tobias Richstein}
 Hier vlt. bisschen auf relevante parts aus related work eingehen + erklären dass wir pretraining machen mit den datensets die dann danach folgen

für image level: ensemble aus yolo + faster rcnn und study level dann ein model 
just a short introduction to our solution, models will be covered in \vref{chapter:detection}
% !TEX root =  master.tex
\chapter{Introduction and problem definition}\label{chapter:intro}

\section{Kaeggle Challenge}
\sectionauthor{Written by Torben}

\section{Related Work}\sectionauthor{Written by Julian}
Exploring the possibilities of Computer Vision methods in medical image analysis reaches back to the late 90's where first proposals of \ac{CAD} systems were designed like in \autocite{kraus2000aluminium}, where the authors detect aluminium dust-induced lung diseases.
With the rise of deep learning, medical image analysis on X-rays is subject of recent academic research reaching from detecting diseases like pneumonia \autocite{pneumoniaDetection} \autocite{pneumoniaDetection2} \autocite{gupta2019evolutionary} to tuberculosis and different thoracic diseases \autocite{jangam2021deep} and even pulmonary \autocite{vieira2021detecting}. 

Even though it is not completely clear how a COVID-19 infection impacts the human body, it is possible to detect typical patterns in lungs using for example chest X-rays of affected patients. This opens many possibilities to provide fast and solid \ac{CAD} solutions while build on knowledge of previous works.

Since the pandemic started in 2020, there are many works published that deal with providing a reliable system for detecting COVID-19 infections using medical images. A \ac{CAD} based process would help the public health sector and relieve medical personal in hospitals by increasing the automation of X-ray examination.
There is already some work proposed in this direction, including the COVID-Net \autocite{wang2020covid} by \citeauthor{wang2020covid}, where they use a machine-driven exploration process to design a deep convolution learning model capable of classifying X-ray images of lungs into three categories (normal, pneumonia, COVID-19). The results of their study look promising and although the authors stated explicitly the non-production-ready state, the work can be used as a basis for future projects.

Another work using a deep neural network is the proposed CovidAID \autocite{mangal2020covidaid} network by \citeauthor{mangal2020covidaid}. In their approach, they use a pre-trained CheXNet \autocite{rajpurkar2017chexnet} \ac{CNN} while substituting the output layer of the model to fit it to their needs of predicting one of the four classes (normal, bacterial pneumonia, viral pneumonia, COVID-19). The authors applied transfer learning by only actively apply the train algorithm to the final layers of their proposed model, whereas the backbone weights are frozen.
In their final comparison, the authors reported to significantly improve upon the results on the previous introduced COVID-Net.
Similar to this the proposed model \autocite{CoronaDLTransfer} by \citeauthor{CoronaDLTransfer} using transformation learning 
Systems that are try to detect positive cases vs. 

using weighted voting ensemble \autocite{livieris2019weighted}

\autocite{brunese2020explainable}

However, since the challenge and the goal set in this project, we need to not only classify input images as COVID-infections, we rather need to detect and locate 
suspicious areas in such images. There is one work proposed going in this direction:

“Fast deep learning computer-aided diagnosis of COVID-19 based on digital chest x-ray images” \autocite{al2021fast}
\section{Proposed Solution}
\sectionauthor{Written by Tobias}
just a short introduction to our solution, models will be convered in \vref{chapter:detection}